\documentclass[a4paper,10pt]{article}
\usepackage[a4paper,margin=2cm]{geometry}
\usepackage{multicol}
\usepackage{fancyhdr}
\usepackage{xcolor}
\usepackage{listings}
\usepackage{titlesec}

% Farben für Code
\definecolor{codebg}{RGB}{245,245,245}
\definecolor{keyword}{RGB}{0,0,180}
\definecolor{comment}{RGB}{0,128,0}
\definecolor{string}{RGB}{163,21,21}

\lstset{
  backgroundcolor=\color{codebg},
  basicstyle=\ttfamily\footnotesize,
  keywordstyle=\color{keyword}\bfseries,
  commentstyle=\color{comment},
  stringstyle=\color{string},
  numbers=left,
  numberstyle=\tiny,
  breaklines=true,
  frame=single,
  showstringspaces=false,
  tabsize=4
}

% Layout für die Überschriften
\titleformat{\section}{\large\bfseries}{}{0em}{}
\titleformat{\subsection}{\normalsize\bfseries}{}{0em}{}

% Kopf- und Fußzeile
\pagestyle{fancy}
\fancyhf{}
\lhead{Bash Cheat Sheet}
\rhead{\thepage}

\begin{document}

\begin{center}
	{\LARGE \textbf{Bash Cheat Sheet}} \\[1em]
	{\large A quick guide to the Bash programming language}
\end{center}

\vspace{0.5cm}

% Section 1: Basics
\section{Basics}
\subsection{Hello, World}
\begin{lstlisting}[language=bash]
#!/bin/bash             # Linux
#!/usr/local/bin/bash   # BSD
#!/usr/bin/env bash     # Portable across systems
echo "Hello World!"
\end{lstlisting}

\subsection{Basic Commands}
\begin{tabbing}
	\= \hspace{30mm} \= \hspace{50mm} \kill
	\> \verb|pwd| \> Shows the current working directory \\
	\> \verb|ls -la| \> Lists all directory contents, including hidden files \\
	\> \verb|cd /path/| \> Changes to the specified directory \\
	\> \verb|clear| \> Clears the terminal screen \\
	\> \verb|man ls| \> Shows documentation for the \verb|ls| command \\
	\> \verb|!!| \> Executes the last command \\
	\> \verb|history| \> Shows a list of previous commands \\
\end{tabbing}

\subsection{Input and Output Terminal}
\begin{tabbing}
	\= \hspace{30mm} \= \hspace{50mm} \kill
	\> \verb|echo "Hallo, $name"| \> Prints Hallo and the value of \verb|name| \\
	\> \verb|read name| \> Reads terminal input into \verb|name| \\
	\> \verb|read -p "Name please: " name| \> Reads terminal input with a prompt \\
	\> \verb|read -s| \> Hidden input, useful for passwords \\
	\> \verb|read -r| \> Prevents interpretation of backslashes \\
\end{tabbing}

\subsection{Data and Directory Manipulation}
\begin{tabbing}
	\= \hspace{60mm} \= \hspace{80mm} \kill
	\> \verb|cp -r /from/path/ /to/path/| \> Copies a directory with its contents \\
	\> \verb|cp /from/path/* /to/path/| \> Copies the contents of a directory \\
	\> \verb|mv /from/path/ /to/path/| \> Moves or renames a directory \\
	\> \verb|mv /from/path/* /to/path/| \> Moves the contents of a directory \\
	\> \verb|rm -rf thing| \> Deletes a file or directory (force) \\
	\> \verb|mkdir bunga| \> Creates the directory \verb|bunga| \\
	\> \verb|touch data.txt| \> Creates an empty file \verb|data.txt| \\
	\> \verb|cat data.txt| \> Shows the content of \verb|data.txt| \\
	\> \verb|less data.txt| \> Views content, better for long files \\
	\> \verb|echo "bunga" > data.txt| \> Writes bunga to the file (overwrites) \\
	\> \verb|echo "bunga" >> data.txt| \> Appends bunga to the file \\
	\> \verb|sed 's/heinz/horst/g' data.txt| \> Replaces all heinz with horst in the file \\
	\> \verb|sed 's/heinz/horst/3' data.txt| \> Replaces the third occurrence of heinz with horst \\
	\> \verb|awk '{print $1}' data.txt| \> Prints the first column of the file \\
	\> \verb|awk '/bunga/ {print}' data.txt| \> Prints lines containing bunga \\
	\> \verb|cut -d "," -f 1 data.csv| \> Extracts the first column from CSV \\
	\> \verb|cut -d ":" -f 1,3 file.txt| \> Extracts columns 1 and 3 (delimiter \:) \\
	\> \verb|cut -c 1-5 file.txt| \> Extracts characters 1 to 5 from each line \\
	\> \verb|cut -d "," -f 2 --complement file.csv| \> Extracts all columns except the second \\
\end{tabbing}

\subsection{Pipe and Redirection}
\begin{tabbing}
	\= \hspace{60mm} \= \hspace{80mm} \kill
	\> \verb|command1 | command2| \> Pipes output of \verb|command1| into \verb|command2| \\
	\> \verb|> | \> Overwrites the file with the new output \\
	\> \verb|>> | \> Appends output to the end of the file \\
	\> \verb|< | \> Reads data as input \\
\end{tabbing}

\subsection{Jobs}
\begin{tabbing}
	\= \hspace{60mm} \= \hspace{80mm} \kill
	\> \verb|jobs| \> Lists jobs running in the background \\
	\> \verb|bg \%1| \> Continues a paused job in the background (e.g., job 1) \\
	\> \verb|fg \%1| \> Brings a background job to the foreground (e.g., job 1) \\
	\> \verb|kill -STOP $PID| \> Stops a job (replace PID with process ID) \\
	\> \verb|kill -CONT $PID| \> Continues a stopped job \\
\end{tabbing}

\subsection{find}
\subsubsection*{Example}
\begin{lstlisting}[language=bash]
#!/bin/bash

# Finding data.txt in /path
find /path -type f -name "data.txt"
\end{lstlisting}

\subsubsection*{Parameters}
\begin{tabbing}
	\= \hspace{60mm} \= \hspace{80mm} \kill
	\> \verb|-type f| \> Searches for files \\
	\> \verb|-type d| \> Searches for directories \\
	\> \verb|-name "bunga"| \> Searches for bunga \\
	\> \verb|-iname "bUnga"| \> Ignores case (e.g., bUnga, BUNGA) \\
	\> \verb|-mtime -n| \> Finds files modified in the last n days \\
	\> \verb|-size +n| \> Finds files larger than n (e.g., +10M for 10MB) \\
	\> \verb|-exec command {} \;| \> Executes a command on found files \\
\end{tabbing}

\subsection{grep}
\begin{tabbing}
	\= \hspace{60mm} \= \hspace{80mm} \kill
	\> \verb|grep "pattern" *.txt| \> Searches for pattern in \.txt files \\
	\> \verb|grep -i "pattern" *.txt| \> Ignores case \\
	\> \verb|grep -r "pattern" /path| \> Searches recursively in /path \\
	\> \verb|grep -n "pattern" *.txt| \> Shows line numbers of matches \\
	\> \verb|grep -l "pattern" *.txt| \> Lists files containing pattern \\
	\> \verb|grep -v "pattern" *.txt| \> Shows lines not matching pattern \\
\end{tabbing}

\subsection{awk}
\begin{tabbing}
	\= \hspace{60mm} \= \hspace{80mm} \kill
	\> \verb|awk '{print $1}' data.txt| \> Prints the first column \\
	\> \verb|awk '/pattern/ {print}' data.txt| \> Prints lines with pattern \\
	\> \verb|awk -F "," '{print $1}' data.csv| \> Prints the first column of a CSV \\
	\> \verb|awk '{print NR, $NF}' file.txt| \> Prints line number and last column \\
\end{tabbing}

\subsection{sort}
\begin{tabbing}
	\= \hspace{60mm} \= \hspace{80mm} \kill
	\> \verb|sort file.txt| \> Sorts lines of \verb|file.txt| alphabetically \\
	\> \verb|sort -r file.txt| \> Sorts in reverse order \\
	\> \verb|sort -n file.txt| \> Sorts numerically \\
	\> \verb|sort -k 2 file.txt| \> Sorts by the second column \\
	\> \verb|sort -t "," -k 1 data.csv| \> Sorts CSV by the first column \\
	\> \verb|sort -u file.txt| \> Sorts and removes duplicate lines \\
\end{tabbing}

\subsection{tar}
\begin{tabbing}
	\= \hspace{30mm} \= \hspace{50mm} \kill
	\> \verb|tar -cvf archive.tar dir/| \> Creates an archive \\
	\> \verb|tar -zcvf archive.tar.gz dir/| \> Creates a gzip archive \\
	\> \verb|tar -xvf archive.tar| \> Extracts an archive \\
	\> \verb|tar -tvf archive.tar| \> Shows the archive contents \\
\end{tabbing}

\subsection{Variables and Arrays}
\begin{tabbing}
	\= \hspace{30mm} \= \hspace{50mm} \kill
	\> \verb|env| \> Shows all environment variables \\
	\> \verb|bunga="1"| \> Creates variable \verb|bunga| with value 1 \\
	\> \verb|bunga=("1" "2")| \> Creates an array \verb|bunga| \\
	\> \verb|echo $bunga| \> Prints the variable (or first array element) \\
	\> \verb|export bunga| \> Makes \verb|bunga| available to subprocesses \\
\end{tabbing}

\subsubsection*{Arrays and Loops}
\begin{tabbing}
	\= \hspace{30mm} \= \hspace{50mm} \kill
	\> \verb|${array[@]}| \> All items in the array \\
	\> \verb|${!array[@]}| \> Iterates through indices \\
	\> \verb|${#array[@]}| \> Length of the array \\
	\> \verb|${array[$i]}| \> Shows item at index \verb|$i| \\
\end{tabbing}

\subsection{Operators}
\subsubsection*{Comparison Operators}
\begin{tabbing}
	\= \hspace{30mm} \= \hspace{50mm} \kill
	\> \verb|-eq| \> Equal to \\
	\> \verb|-ne| \> Not equal to \\
	\> \verb|-lt| \> Less than \\
	\> \verb|-le| \> Less than or equal to \\
	\> \verb|-gt| \> Greater than \\
	\> \verb|-ge| \> Greater than or equal to \\
\end{tabbing}

\subsubsection*{String Comparison Operators}
\begin{tabbing}
	\= \hspace{30mm} \= \hspace{50mm} \kill
	\> \verb|=| \> Equal to \\
	\> \verb|!=| \> Not equal to \\
	\> \verb|<| \> Less than (ASCII order) \\
	\> \verb|>| \> Greater than (ASCII order) \\
\end{tabbing}

\subsubsection*{Arithmetic Operators}
\begin{tabbing}
	\= \hspace{30mm} \= \hspace{50mm} \kill
	\> \verb|+| \> Addition \\
	\> \verb|-| \> Subtraction \\
	\> \verb|*| \> Multiplication \\
	\> \verb|/| \> Division \\
	\> \verb|\%| \> Modulus (remainder of division) \\
\end{tabbing}

\subsubsection*{Logical Operators}
\begin{tabbing}
	\= \hspace{30mm} \= \hspace{50mm} \kill
	\> \verb|&&| \> Logical AND \\
	\> \verb||| \> Logical OR \\
	\> \verb|!| \> Logical NOT \\
\end{tabbing}

\subsubsection*{File Operators}
\begin{tabbing}
	\= \hspace{30mm} \= \hspace{50mm} \kill
	\> \verb|-e| \> Checks if file exists \\
	\> \verb|-d| \> Checks if directory exists \\
	\> \verb|-s| \> Checks if file is not empty \\
\end{tabbing}

\subsection{Loops}
\subsubsection*{for}
\begin{lstlisting}[language=bash]
#!/bin/bash
for i in {1..5}; do
  echo $i
done

# Arrays with items
for item in "${array[@]}"; do 
  echo "$item"
done 

# Arrays with indices
for i in "${!array[@]}"; do 
  echo "Index $i: ${array[$i]}"
done
\end{lstlisting}

\subsubsection*{while}
\begin{lstlisting}[language=bash]
#!/bin/bash
count=2

while [ $count -le 5 ]; do
  echo $count
  ((count++))
done

# Arrays
i=0 
while [ $i -lt ${#array[@]} ]; do 
  echo "${array[$i]}"
  ((i++))  % Fixed: Corrected {{i++}} to ((i++))
done
\end{lstlisting}

\subsubsection*{until}
\begin{lstlisting}[language=bash]
#!/bin/bash
count=1
until [ $count -gt 5 ]; do
  echo $count
  ((count++))
done

# Arrays 
i=0 
until [ $i -ge ${#array[@]} ]; do 
  echo "${array[$i]}"
  ((i++))  % Fixed: Corrected {{i++}} to ((i++))
done
\end{lstlisting}

\subsubsection*{break and continue}
\begin{lstlisting}[language=bash]
#!/bin/bash
for i in {1..5}; do
  if [ $i -eq 3 ]; then
    continue
  fi
  echo "Number $i"
  if [ $i -eq 4 ]; then
    break
  fi
done
\end{lstlisting}

\subsection{if}
\begin{lstlisting}[language=bash]
#!/bin/bash
num=10
if [ $num -gt 10 ]; then
  echo "Number is greater than 10"
elif [ $num -eq 10 ]; then
  echo "Number is exactly 10"
else
  echo "Number is less than 10"
fi
\end{lstlisting}

\subsection{case}
\begin{lstlisting}[language=bash]
#!/bin/bash
case $variable in
  pattern1) commands ;; 
  pattern2) commands ;; 
  *) default_commands ;;
esac
\end{lstlisting}

\subsection{Exit Codes}
\begin{lstlisting}[language=bash]
#!/bin/bash
if [ "$1" == "test" ]; then  % Fixed: Added missing [
  exit 0 
else 
  exit 1  # 1 to 255
fi
echo $?  # Shows the last exit code
\end{lstlisting}

\subsection{Error Handling}
\begin{tabbing}
	\= \hspace{30mm} \= \hspace{50mm} \kill
	\> \verb|set -e| \> Ends script on any error \\
	\> \verb|set +e| \> Deactivates ending on errors \\
	\> \verb|set -u| \> Treats undefined variables as an error \\
	\> \verb|set -o pipefail| \> Makes \$? reflect the last failed pipe command \\
	\> \verb|set +o pipefail| \> Deactivates pipefail \\
	\> \verb|2> error.log| \> Writes errors to \verb|error.log| \\
	\> \verb|set -x| \> Activates debugging \\
	\> \verb|set +x| \> Deactivates debugging \\
	\> \verb|command 2> /dev/null| \> Suppresses error messages \\
\end{tabbing}

\subsection{Logging}
\begin{tabbing}
	\= \hspace{30mm} \= \hspace{50mm} \kill
	\> \verb|/var/log/| \> Common path for Linux logs (requires root) \\
	\> \verb|logger "message"| \> Writes message to system log \\
\end{tabbing}

\subsubsection*{Custom Logging Example}
\begin{lstlisting}[language=bash]
echo "$(date '+%Y-%m-%d %H:%M:%S') - message" >> ~/logs/logfile.log
\end{lstlisting}

\subsection{Functions}
\begin{lstlisting}[language=bash]
#!/bin/bash
my_function() {
  echo "Hello World!"
}
my_function
\end{lstlisting}

\subsection{alias}
\begin{lstlisting}[language=bash]
#!/bin/bash
# Like functions, but simpler; no parameters, often used to combine commands
alias myfunc='ls -la'
\end{lstlisting}

\end{document}
