\documentclass[a4paper,12pt,twoside]{article}
\usepackage[ngerman]{babel}
\usepackage[utf8]{inputenc}
\usepackage[T1]{fontenc}
\usepackage{geometry}
\usepackage{fancyhdr}
\usepackage[hidelinks]{hyperref}
\usepackage{listings}
\usepackage{xcolor}
\usepackage{ulem}
\usepackage{graphicx}
\usepackage{amsmath}
\usepackage{hyperref}
\usepackage{caption}
\usepackage{siunitx}

\geometry{
  a4paper, 
  left=3cm,
  right=3cm,
  top=2cm,
  bottom=2cm
}

% Farben für Code
\definecolor{codebg}{RGB}{245,245,245}
\definecolor{keyword}{RGB}{0,0,180}
\definecolor{comment}{RGB}{0,128,0}
\definecolor{string}{RGB}{163,21,21}

\lstset{
  backgroundcolor=\color{codebg},
  basicstyle=\ttfamily\footnotesize,
  keywordstyle=\color{keyword}\bfseries,
  commentstyle=\color{comment},
  stringstyle=\color{string},
  numbers=left,
  numberstyle=\tiny,
  breaklines=true,
  frame=single,
  showstringspaces=false,
  tabsize=4
}

% Titelseite
\title{Programmiersprache C}
\author{Tobi}
\date{\today}

% Kopf und Fußzeilen
\pagestyle{fancy}
\fancyhf{} % Kopf- und Fußzeilen leeren
\fancyfoot[C]{\thepage} % Seitenzahl zentriert in der Fußzeile
\fancyhead[LE, RO]{\title}
\fancyhead[LO, RE]{\author}


\begin{document}

% Titelseite
\maketitle
\thispagestyle{empty} % Keine Kopf-/Fußzeile auf der Titelseite
\newpage

% Inhaltsverzeichnis
\tableofcontents
\newpage

\section{Variablen}
\subsection{Namen}
Zulässige Namen enthalten:
\begin{itemize}
  \item Buchstaben a \- z und A \- Z 
  \item Zahlen 0 \- 9 
  \item Unterstrich \_ 
  \item Name darf nicht mit einer Zahl beginnen 
  \item es wird zwischen Groß\- und Kleinschreibung unterschieden 
  \item dürfen keine Schlüsselwörter enthalten wie if, for, while usw.
  \item keine Beschränkung der Länge 
  \item die ersten 31 Ziffern müssen sich unterscheiden
\end{itemize}

\section{Datentypen} 
\subsection{Ganze Zahlen}
\begin{center}
	\begin{tabular}{|c|c|c|c|}
    \hline
    Typ & Größe & unsigned & signed \\ 
    \hline 
    \hline 
    short & 2 Byte & 0 bis $2^{16}-1$ & $-2^{15}$ bis $2^{15}-1$ \\ 
    \hline
    int & 2 Byte & 0 bis $2^{16}-1$ & $-2^{15}$ bis $2^{15}-1$ \\ 
    int & 4 Byte & 0 bis $2^{32}-1$ & $-2^{31}$ bis $2^{31}-1$ \\
    \hline
    long & 4 Byte & 0 bis $2^{32}-1$ & $-2^{31}$ bis $2^{31}-1$ \\
    long & 8 Byte & 0 bis $2^{64}-1$ & $-2^{63}$ bis $2^{63}-1$ \\
    \hline 
  \end{tabular} 
\end{center}
Wenn kein Modifikator (unsigned/signed) geschrieben, wird signed verwendet.
\\
\begin{center}
  \begin{tabular}{|c|c|}
    \hline 
    Suffix & unsigned int bunga = 238u \\
    \hline 
    \hline
    u & unsigned \\  
    \hline
    ul & unsigned long \\
    \hline
    ll & long long \\ 
    \hline
    ull & unsigned long long \\
    \hline
  \end{tabular}
\end{center}
Ohne Suffix, schaut der Compiler was er verwenden muss, mit \verb|unsigned int bunga = 238u| \\ 
wird der Compiler gezwungen unsigned zu verwenden. \\ 
Es könnten Fehler passieren, wie: \\ 
\verb| unsigned int big = 3000000000u; // 3 Milliarden, passt in unsigned int, aber nicht in int| \\

\subsection{Gleitkommazahl}
\begin{center}
  \begin{tabular}{|c|c|c|c|c|c|}
    \hline 
    Typ & Größe & Wertebereich & Nachkommastellen & Exponent & Mantisse \\ 
    \hline 
    \hline 
    float & 4 Byte &\(\pm 1.18 \times 10^{-38}\) bis \(\pm 3.4 \times 10^{38}\) & \textasciitilde 7 & 8 Bit & 23 Bit \\ 
    \hline 
    double & 8 Byte &\(\pm 2.23 \times 10^{-308}\) bis \(\pm 1.79 \times 10^{308}\) & \textasciitilde 15 & 11 Bit & 52 Bit \\ 
    \hline 
    long double & 8 Byte &\(\pm 2.23 \times 10^{-308}\) bis \(\pm 1.79 \times 10^{308}\) & \textasciitilde 15 & 11 Bit & 52 Bit \\ 
    \hline 
    long double & 10 Byte &\(\pm 3.37 \times 10^{-4932}\) bis \(\pm 1.18 \times 10^{4932}\) & \textasciitilde 19 & 15 Bit & 64 Bit \\ 
    (Boarland) & & & & & \\
    \hline 
  \end{tabular}
\end{center}

\begin{center}
  \begin{tabular}{|c|c|}
    \hline 
    Suffix & \\ 
    \hline 
    \hline 
    float & f \\
    \hline 
    long double & L \\ 
    \hline
  \end{tabular}
\end{center}
\end{document}
