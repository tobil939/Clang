\documentclass[a4paper,10pt]{article}
\usepackage[a4paper,margin=2cm]{geometry}
\usepackage{multicol}
\usepackage{fancyhdr}
\usepackage{xcolor}
\usepackage{listings}
\usepackage{titlesec}

% Farben für Code
\definecolor{codebg}{RGB}{245,245,245}
\definecolor{keyword}{RGB}{0,0,180}
\definecolor{comment}{RGB}{0,128,0}
\definecolor{string}{RGB}{163,21,21}

\lstset{
  backgroundcolor=\color{codebg},
  basicstyle=\ttfamily\footnotesize,
  keywordstyle=\color{keyword}\bfseries,
  commentstyle=\color{comment},
  stringstyle=\color{string},
  numbers=left,
  numberstyle=\tiny,
  breaklines=true,
  frame=single,
  showstringspaces=false,
  tabsize=4
}

% Layout für die Überschriften
\titleformat{\section}{\large\bfseries}{}{0em}{}
\titleformat{\subsection}{\normalsize\bfseries}{}{0em}{}

% Kopf- und Fußzeile
\pagestyle{fancy}
\fancyhf{}
\lhead{C/C++ Cheat Sheet}
\rhead{\thepage}

\begin{document}

\begin{center}
	{\LARGE \textbf{C/C++ Cheat Sheet}} \\[1em]
	{\large A quick guide to C and C++ programming}
\end{center}

\vspace{0.5cm}

\section{Basics}
\subsection{Hello, World}
\begin{lstlisting}[language=C]
#include <stdio.h>  // C
int main() {
    printf("Hello World!\n");
    return 0;
}
// C++:
#include <iostream>
int main() {
    std::cout << "Hello World!" << std::endl;
    return 0;
}
\end{lstlisting}

\subsection{Basic Commands/Functions}
\begin{tabbing}
	\= \hspace{30mm} \= \hspace{50mm} \kill
	\> \verb|printf("text");| \> Prints formatted text (C) \\
	\> \verb|std::cout| \> Prints text (C++) \\
	\> \verb|scanf("\%d", &var);| \> Reads input (C) \\
	\> \verb|std::cin >> var;| \> Reads input (C++) \\
	\> \verb|system("cls");| \> Clears console (platform-dependent) \\
\end{tabbing}

\subsection{Input and Output}
\begin{tabbing}
	\= \hspace{30mm} \= \hspace{50mm} \kill
	\> \verb|printf("\%d", var);| \> Outputs integer (C) \\
	\> \verb|std::cout << var;| \> Outputs variable (C++) \\
	\> \verb|scanf("\%s", str);| \> Reads string (C, no for arrays) \\
	\> \verb|std::cin >> str;| \> Reads string (C++) \\
\end{tabbing}

\subsection{Variables and Data Types}
\begin{tabbing}
	\= \hspace{30mm} \= \hspace{50mm} \kill
	\> \verb|int x = 10;| \> Integer \\
	\> \verb|float y = 5.5;| \> Floating point \\
	\> \verb|char c = 'a';| \> Character \\
	\> \verb|bool b = true;| \> Boolean (C++ only) \\
	\> \verb|std::string s = "text";| \> String (C++ only) \\
\end{tabbing}

\subsection{Arrays and Pointers}
\begin{tabbing}
	\= \hspace{30mm} \= \hspace{50mm} \kill
	\> \verb|int arr[5] = {1,2,3,4,5};| \> Array initialization \\
	\> \verb|int *ptr = arr;| \> Pointer to array \\
	\> \verb|*ptr| \> Dereferences pointer \\
	\> \verb|ptr + 1| \> Points to next element \\
\end{tabbing}

\subsection{Operators}
\subsubsection*{Arithmetic}
\begin{tabbing}
	\= \hspace{30mm} \= \hspace{50mm} \kill
	\> \verb|+| \> Addition \\
	\> \verb|-| \> Subtraction \\
	\> \verb|*| \> Multiplication \\
	\> \verb|/| \> Division \\
	\> \verb|\%| \> Modulus \\
\end{tabbing}

\subsubsection*{Comparison}
\begin{tabbing}
	\= \hspace{30mm} \= \hspace{50mm} \kill
	\> \verb|==| \> Equal to \\
	\> \verb|!=| \> Not equal to \\
	\> \verb|<| \> Less than \\
	\> \verb|>| \> Greater than \\
\end{tabbing}

\subsection{Loops}
\subsubsection*{for}
\begin{lstlisting}[language=C]
for (int i = 0; i < 5; i++) {
    printf("%d\n", i);
}
// C++:
for (int i : {1,2,3,4,5}) {  // Range-based (C++11)
    std::cout << i << std::endl;
}
\end{lstlisting}

\subsubsection*{while}
\begin{lstlisting}[language=C]
int i = 0;
while (i < 5) {
    printf("%d\n", i);
    i++;
}
\end{lstlisting}

\subsubsection*{do-while}
\begin{lstlisting}[language=C]
int i = 0;
do {
    printf("%d\n", i);
    i++;
} while (i < 5);
\end{lstlisting}

\subsection{Conditionals}
\subsubsection*{if}
\begin{lstlisting}[language=C]
int x = 10;
if (x > 5) {
    printf("Greater than 5\n");
} else if (x == 5) {
    printf("Equal to 5\n");
} else {
    printf("Less than 5\n");
}
\end{lstlisting}

\subsubsection*{switch}
\begin{lstlisting}[language=C]
int x = 2;
switch (x) {
    case 1: printf("One\n"); break;
    case 2: printf("Two\n"); break;
    default: printf("Other\n");
}
\end{lstlisting}

\subsection{Functions}
\begin{lstlisting}[language=C]
// C:
int add(int a, int b) {
    return a + b;
}
// C++:
int add(int a, int b) {
    return a + b;
}
int main() { printf("%d\n", add(2, 3)); }
\end{lstlisting}

\subsection{Pointers and Memory}
\begin{tabbing}
	\= \hspace{30mm} \= \hspace{50mm} \kill
	\> \verb|&var| \> Address of variable \\
	\> \verb|*ptr| \> Value at address \\
	\> \verb|malloc(size);| \> Allocates memory (C) \\
	\> \verb|free(ptr);| \> Frees memory (C) \\
	\> \verb|new int;| \> Allocates memory (C++) \\
	\> \verb|delete ptr;| \> Frees memory (C++) \\
\end{tabbing}

\subsection{Error Handling}
\begin{tabbing}
	\= \hspace{30mm} \= \hspace{50mm} \kill
	\> \verb|if (ptr == NULL)| \> Check for null pointer (C) \\
	\> \verb|try { ... } catch (...) { ... }| \> Exception handling (C++ only) \\
	\> \verb|errno| \> Error number (C) \\
	\> \verb|perror("Error");| \> Prints error message (C) \\
\end{tabbing}

\subsection{File I/O}
\begin{tabbing}
	\= \hspace{30mm} \= \hspace{50mm} \kill
	\> \verb|FILE *f = fopen("file.txt", "r");| \> Opens file (C) \\
	\> \verb|fclose(f);| \> Closes file (C) \\
	\> \verb|std::ifstream file("file.txt");| \> Opens file (C++) \\
	\> \verb|file.close();| \> Closes file (C++) \\
\end{tabbing}\

\subsection{Classes (C++ only)}
\begin{lstlisting}[language=C++]
class MyClass {
public:
    int x;
    MyClass(int val) : x(val) {}
    void print() { std::cout << x << std::endl; }
};
int main() {
    MyClass obj(5);
    obj.print();
}
\end{lstlisting}

\end{document}
